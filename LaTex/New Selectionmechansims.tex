\documentclass[a4paper , 11pt]{article}
%\usepackage{booktabs}

\usepackage[utf8]{inputenc}
\usepackage{natbib}
\usepackage{booktabs}
%\usepackage[latin1]{inputenc}
\usepackage{eurosym}
\usepackage[ngerman,english]{babel}
\usepackage{latexsym}
\usepackage{ulem}
\usepackage{array}
\usepackage{amssymb}
\interfootnotelinepenalty=10000
\usepackage[letterpaper,left=2cm,right=2cm,top=2cm,bottom=2cm]{geometry}

\usepackage{epsfig}
\usepackage[doublespacing]{setspace}
\usepackage{epsfig}
\usepackage{lscape}

\usepackage{latexsym}
\usepackage{amsmath}
\usepackage{multirow}
\usepackage{setspace}
\usepackage{ctable}
\usepackage{url}
\usepackage[english]{babel}
\usepackage{eurosym}
\usepackage{latexsym}
\usepackage{url}

%\MakeOuterQuote{"}
%\usepackage[backend=biber, style=authoryear, natbib=true, backref=true, maxbibnames=99, doi=false, url=false]{biblatex}
%\addbibresource{quellen.bib}

\title{Alternative Selectionmechanisms in Online Samples}

\author  {Franz Prücklmair\thanks{Universität Bamberg} \\ Ulrich Rendtel\thanks{FB Wirtschaftswissenschft, Freie Universität Berlin} 
        }

\date{ \today\\Manuscript under construction\\ Do not cite}

\begin{document}

\maketitle
\tableofcontents

\begin{abstract}

Online-Surveys gehören aufgrund der relativ günstigen und schnellen Umsetzung zu den weitverbreitetsten Datenerhebungsmethoden. Allerdings weisen die daraus gewonnenen Daten, sogenannte Non-probability Samples einige Probleme auf. Dazu gehören mögliche Verzerrungen durch Selektivität, also des systematischen Ausschlusses von Personengruppen aus der Grundgesamtheit. Auch Erhebungsmethoden, wie z.B. River-Sampling, bei dem Befragte oftmals durch Implementation von Widgets auf Webseiten unkontrolliert rekrutiert werden, führen dazu, dass die für die Gewichtung benötigte inclusion probability des Befragten unbekannt ist. Somit ist die klassische design-basierende Inferenz wie im Probability Sample für Non-probability Samples unmöglich.
Der Ansatz der Quasi-randomisation versucht durch Kombination eines Online-Samples mit einem Probability (Reference) Sample den Selektionsmechanismus zu modellieren und mithilfe von Pseudogewichtung die Verzerrung zu korrigieren. Ein häufig modellierter Selektionsmechanismus ist bislang der Internetzugang. Mit fortschreitendem Netzausbau ist jedoch davon auszugehen, dass der Ausschluss von Personen ohne Internetzugang immer mehr vernachlässigbar wird. Es könnte sich daher lohnen, den QR-Ansatz unter der Modellierung alternativer Selektionsmechanismen zu untersuchen.
Für Analysen dient die 9. Runde des European-Social-Survey (ESS) aus dem Jahr 2018 als Datengrundlage. Dieser übernimmt im QR-Ansatz die Rolle als klassisches Reference Sample und dient gleichzeitig, auf Basis alternativer Selektionsmechanismen wie der „Mindestdauer im Internet“ oder der „Bereitschaft, digitale Inhalte zu verbreiten“, als Quelle für die Simulation der Online-Samples. Die daraus resultierenden pseudo-gewichteten Ergebnisse zur Bundestagswahl 2017 sollen anhand der vom ESS geschätzten und dem tatsächlichen Wahlergebnis verglichen werden. Zudem sollen die Korrekturen mit klassischen Kalibrationsmethoden wie dem Raking gegenübergestellt werden.
Erste Ergebnisse zeigen, dass der klassische Internetzugang nicht mehr selektiv ist. Dies spricht für die Verwendung alternativer Selektionsmechanismen, um den QR-Ansatz zu evaluieren. Die alternativen Modelle zeigen zudem, dass sich starke Verzerrungen in den Wahlergebnissen teilweise sehr gut korrigieren lassen. Gleichzeitig können durch den Ansatz auch vergleichbare Grenzen bei der Korrektur von Einstellungen und Meinungen deutlich gemacht werden. 

\begin{center}
  \textbf{Abstract:} Keywords: Sampling frame, Access Panel, River Sampling, Non-probability samples
\end{center}



\end{abstract}
\newpage
%%%%%%%%%%%%%%%%%%%%%%%%%%%%%%%%%%%%%%%%%%%%%%%%%%%%%%%%%%%%%%%%%%%%%%%%%%%%%%%%%%%%%%%%%%%%%%%%%%
%%%%%%%%%%%%%%%%%%%%%%%%%%%%%%%%%%%%%%%%%%%%%%%%%%%%%%%%%%%%%%%%%%%%%%%%%%%%%%%%%%%%%%%%%%%%%%%%%%
\section{Online surveys und Ihre Probleme}
Allgemein Haben Online Surveys als sogenannte Nonüprobability samples Probleme
Welche Wieso? usw. Es gibt aber unterschiedliche Arten von Online Surveys
\subsection{Arten von Online Surveys }
Classical sampling and change to omlöine mode
Sampling via Mail
Sampling from Accespanel
River sampling via Widgets

Entscheidend hiebi ist vorallam der Selektionsmechanismus

\section{Selecktionmechanisms und Online Samples und Korrektur verfahren}
\subsection{Selektionsmechanismen }
Internet Zugang nicht mehr aussage Kräftig neuer Selektionsmechanismen entscheidend
\subsection{Quasirandomisation}
\subsection{Wahlumfragen und ihre aussagekraft}

\section{Method}
\subsection{Datenstz der ESS}
\subsection{Simulationaufbau}
\section{Results}
ESS Vergleich
SAR
Abs Bias Parteipräferenz
Rel Bias Partei präf
Boxplots
Gewichts betrachtung
\section{Discussion}
Netzzugang nicht mehr aussagekräftig
Korrektrur funktioniert bei etablirten Parteien sehr gut bei kleineren weniger.
Messungen von Einstelllungen schwierig
Gilt für alle Selektionsmechanismen 
Alternative Selektionsmechanismen Identifizieren


 \end{document}

