\documentclass[a4paper]{article}
%\usepackage{simplemargins}

%\usepackage[square]{natbib}
\usepackage{amsmath}
\usepackage{amsfonts}
\usepackage{amssymb}
\usepackage{graphicx}

\begin{document}
\pagenumbering{gobble}

\Large
 \begin{center}
New Selection Mechanisms in Online Surveys\\ 

\hspace{10pt}

% Author names and affiliations
\large
Franz Prücklmair$^1$ Ulrich Rendtel$^2$ \\

\hspace{10pt}

\small  
$^1$) Otto-Friedrich-Universität Bamberg\\
franz.pruecklmair@uni-bamberg.de\\
$^2$) Freie Universität Berlin

\end{center}

\hspace{10pt}

\normalsize

Despite their well-documented difficulties, online surveys are an integral part of data collection. There are many reasons for this, the most common are low cost and simplicity of implementation. Also, technical development may enable higher response rates, such as access to real-time results or clever implementation of survey gadgets on websites. Larger companies are taking this development into account by using different techniques such as river sampling to obtain data.
\\

On the other hand, web surveys as so-called non-probability samples are also problematic. It can be assumed, that the respondents recruited on the internet cannot be transferred simply to the total population. Since this selection process is mostly out of the control of the researcher and therefore unkown, a classical design-based approach of the web survey is almost impossible. There are some approaches, which try to take this problem into account and allow inference by weighting, matching or using superpopulations. It should be considered, that many evaluations of approaches use only internet access as selection mechanism. However, it can be assumed, that this selection is subject to strong changes in the course of an increasing network expansion. 
\\

We therefore propose a new approach that uses alternative selection mechanisms to evaluate the proven correction methods. To this end, the 2018 ESS was used to examine the quasi-randomization and calibration approach under three different selection mechanisms, namely: duration of internet use, internet activity, and a combination of topic interest and internet duration. The results show on the one hand the limits of the selectivity of the internet access and on the other hand, they also show differences in performance under the chosen selection mechanisms.  Finally, the weighted results of the non-probability can be compared with the classical survey but also with the actual election results.ö

\end{document}
