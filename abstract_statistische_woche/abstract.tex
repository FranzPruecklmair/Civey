\documentclass[a4paper]{article}
%\usepackage{simplemargins}

%\usepackage[square]{natbib}
\usepackage{amsmath}
\usepackage{amsfonts}
\usepackage{amssymb}
\usepackage{graphicx}

\begin{document}
\pagenumbering{gobble}

\Large
 \begin{center}
Neue Selektionsmechanismen in Online-Umfragen\\ 

\hspace{10pt}

% Author names and affiliations
\large
Franz Prücklmair$^1$ Ulrich Rendtel$^2$ \\

\hspace{10pt}

\small  
$^1$) Otto-Friedrich-Universität Bamberg\\
franz.pruecklmair@uni-bamberg.de\\
$^2$) Freie Universität Berlin

\end{center}

\hspace{10pt}

\normalsize

Online-Umfragen sind trotz ihrerbekannten Problemen, immer noch fester Besandteil der Datenerhebung. Eine schnelle und kostengünstige Handhabung führt dazu, dass Unternehmen immer noch verstärktt auf Online-rekrutierungen setzen. Auch neuere technische Entwicklungen wie ein verbreiteter Netztausbau oder visuelle ansprechende digitale Gestaltung können als Beispiele dienen, warum die Bereitwilligkeit an Online-Umfragen teilzunehmen, gestiegen ist.
\\


Andererseits ist das Arbeiten mit Webumfragen von zahlreichen Herausforderungen geprägt, von einer unbekannten Teilnahmewahrscheinlichkeit bis hin zu einem kompletten Auschluss bestimmter (Offline)-Gruppen. Dies macht eine klassische Inferenz wie im Probabiliy-Sampling nahezu unmöglich. Es gibt jedoch einige Methoden, welche versuchen diese Herausforderungen durch unterschiedliche Ansätze zu korrigieren. Eine Methode  ist der Ansatz der Quasi-randomisierung, mithilfe dessen sogenannte Pseudo-Gewichte erzeugt werden, welche eine Teilnahmewahrscheinlichkeit simulieren sollen. In der Vergangenheit wurde wurde das Potential dieses Ansatzes bereits auf die Korrektur des Bias durch den Auschluss von Befragten ohne Internet verursacht, untersuchgt. Allerdings ist davon auszugehen, dass Ergebnisse durch einen verbreiteten Netztausbau nicht mehr aktuell  sind und eventuell neuere Mechanismen, ein aktuelleres Bild für eine Digitale-Teilnahme liefern.
\\

Wir wollen daher den Quasirandomisation-Ansatz unter Verwendung des ESS von 2018 erneut untersuchen. Und die Evaluation dieses Ansatzes mithilfe unterschiedlicher alternativen Selektionsmechanismen analysieren. Hierbei soll der Initierte Bias des klassische Selektionsmechanismus "Internetzugang" untersucht werden und dieser im Vergleich zur neueren Selektionsmechanismen wie zum Beispiel "verbrachte Zeit im Internet" oder bestimmte Formen von Netzaktivität betrachtet werden. Es soll anschliesend gezeigt werden, ob der Quasirandomisierungsansatz es schafft, diesen Bias zu korrigieren und diesen im Vergleich zu quantifizieren. 



\end{document}